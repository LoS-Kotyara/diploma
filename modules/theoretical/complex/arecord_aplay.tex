\subsubsection{Утилиты для записи и воспроизведения звука}

Arecord -- это утилита для командной строки, предназначенная для записи звуковых данных со звуковой подсистемы ALSA. Она поддерживает несколько форматов записи данных, которые совпадают с форматами, поддерживаемыми ALSA, и одновременную запись с нескольких устройств~\cite{arecord}.

Aplay представляет собой обратную arecord утилиту. С помощью aplay можно воспроизводить заранее записанные звуковые дорожки. Утилита также поддерживает все форматы, поддерживаемые ALSA~\cite{aplay}.

При работе как с arecord, так и с aplay, необходимо указать~\cite{arecord, aplay}:

\begin{enumerate}
  \item Формат выходного/входного файла (по умолчанию wav).
  \item Формат данных, или битовая глубина (по умолчанию U8 -- беззнаковое 8-битное).
  \item Количество каналов (по умолчанию 1).
  \item Устройство, с которого происходит запись, или на котором звуковая дорожка должна проигрываться (по умолчанию устройство, указанное в файле конфигурации ALSA).
  \item Частоту дискретизации (по умолчанию 8 кГЦ).
  \item Файл для вывода/ввода данных.
\end{enumerate}

Полученные arecord звуковые дорожки, в случае формата файла RAW, частоте дискретизации 8 кГц, использовании одного канала и формате данных U8, представляют собой 8000 записей со значениями от 0 до 255 каждую секунду.

Если необходимо конвейерно использовать данные утилиты arecord, необходимо использовать перенаправление потоков. То есть перенаправить стандартный вывод утилиты arecord в стандартный вход следующей команды. Для этого вместо имени выходного файла необходимо использовать знак тире (<<->>), и в следующей за ней команде так же использовать знак тире. Например, чтобы перенаправить данные с микрофона на наушники, используя утилиты arecord и aplay, необходимо использовать следующий скрипт:

\begin{lstlisting}[style=ES6, language=bash]
  #!/bin/bash
  arecord -t raw -f cd | aplay - -t raw -f cd
\end{lstlisting}
