\subsubsection{Advanced Linux Sound Architecture, PCM, RAW}

Для обработки звука в Linux-системах применяются два вида звуковых подсистем:

\begin{enumerate}
  \item ALSA -- Advanced Linux Sound Architecture, Продвинутая звуковая архитектура Linux.
  \item OSS -- Open Sound System, Открытая звуковая система.
\end{enumerate}

OSS использовался в Linux ядре ветки 2.4. Но, из-за наличия закрытого кода и платной лицензии, что контрастирует с названием системы, был заменён в последующих версиях ядра на ALSA.

ALSA предоставляет функциональность аудио драйвера и цифрового интерфейса музыкальных инструментов в Linux. ALSA тесно связана с ядром Linux. ALSA -- программный микшер, который эмулирует совместимость для других слоёв. Также предоставляет API для программистов и работает с низкой и стабильной задержкой~\cite{alsa-project}. ALSA обладает следующими ключевыми особенностями:

\begin{enumerate}
  \item Эффективная поддержка всех типов звуковых интерфейсов, от любительских до профессиональных многоканальных интерфейсов.
  \item Полностью модульные звуковые драйвера.
  \item Аппаратное микширование нескольких каналов.
  \item Полнодуплексная работа.
  \item Поддерживающие многопроцессорность и драйверы с потоковой безопасностью, thread-safe драйверы.
  \item Открытая библиотека для упрощения разработки программного обеспечения и обеспечения высокоуровневого функционала.
  \item Поддержка более старого OSS API, обеспечение бинарной совместимости для большинства OSS программ.
\end{enumerate}

Наиболее часто для оцифровки аналоговых звуковых сигналов используется импульсно-кодовая модуляция~\cite{alsa-arch}. Данные, поступающие на записывающее устройство, подключенное к цифровому интерфейсу, кодируются с помощью PCM на звуковой карте, и затем передаются на звуковую подсистему ALSA.

При использовании PCM, аналоговый передаваемый сигнал преобразуется в цифровую форму посредством трёх операций: дискретизации по времени, квантования по амплитуде и кодирования. Сначала фиксируется амплитуда сигнала через определённые промежутки времени и регистрируются полученные значения амплитуды в виде округлённых цифровых значений. Затем полученные данные кодируются и передаются в канал связи.

Данные, полученные каналом связи, ALSA может представлять в 4 форматах файлов:

\begin{enumerate}
  \item VOC -- содержит заголовок, звуковые данные в сжатом виде.
  \item WAV -- содержит заголовок, звуковые данные не в сжатом виде.
  \item RAW -- не содержит заголовок, звуковые данные не в сжатом виде.
  \item AU -- содержит заголовок и аннотацию, звуковые данные в сжатом виде.
\end{enumerate}

При обработке звуковых данных, наиболее удобно использовать формат RAW. В этом случае, данные сохраняются без сжатия и заголовков, данные представлены в виде PCM-кодов.