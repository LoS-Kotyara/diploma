\subsubsection{Запись и обработка звука}

Для запуска команд arecord и aplay, предоставляющих возможность записи и воспроизведения звуков, необходимо создание дочерних процессов, управляемых из основного процесса сервера.

Node обладает встроенным модулем <<child\_process>>, предоставляющий возможность запуска, завершения и управления дочерними процессами. Существует 4 способа запуска дочерних процесса~\cite{node}:

\begin{itemize}
  \item child\_process.exec() -- exec;
  \item child\_process.execFile() -- execFile;
  \item child\_process.fork() -- fork;
  \item child\_process.spawn() -- spawn.
\end{itemize}

Exec и execFile используются, когда необходим запуск команды или файла, содержащего команды, или исполняемого файла, без возможности управления его стандартными потоками. В этом случае создаётся отдельный процесс, выполняемый, пока на одном из стандартных потоков не появятся данные. Например, эти команды можно использовать для запуска какой-либо системной службы или получения информации о файле.

Spawn используется для запуска команды с некоторыми параметрами запуска. В этом случае возможно управление стандартными потоками ввода, вывода и ошибок. Также, процесс, запущенный с помощью spawn будет активен, пока не получит терминальный сигнал от родительского процесса или от операционной системы.

Fork является подвидом spawn, предоставляющим возможность запуска нового Node процесса.

Для запуска процессов, выполняющих команды aplay и arecord, был использован способ spawn.

Например, используя скрипт, представленный в листинге~\ref{lst:node__spawn_pipe}, можно запустить процессы, выполняющие запись и воспроизведение звука, перенаправив выходной стандартный поток процесса записи во входной стандартный поток процесса воспроизведения.

\begin{lstlisting}[style=ES6, caption={Пример запуска процессов и перенаправления потоков процессов}, label={lst:node__spawn_pipe}]
const arecordProcess = spawn('arecord', [
    '-t', 'raw'
]);

const aplayProcess = spawn('aplay', ['-',
  '-t', 'raw'
]);

arecordProcess.stdout.pipe(aplayProcess.stdin);
\end{lstlisting}

Дочерний процесс, в котором выполняется команда arecord, в случае работы в одноканальном режиме с частотой дискретизации 8000 Гц и битовой глубиной 8 бит, возвращает в стандартный поток вывода каждую секунду 8000 записей со значениями от 0 до 255, показывающими уровень шума в соответствующий момент времени.

Для преобразования полученного звукового потока в световой, необходимо разработать и реализовать процесс обработки звука.

Был выбран следующий способ обработки:

\begin{itemize}
  \item Каждый интервал времени t1 получаются значения, полученные после импульсно-кодовой модуляции звука, и выбирается максимальный элемент max1, а именно его значений и его индекс.
  \item Проверяется, является ли полученное значение максимального элемента max1 большим чем некоторый порог. Если значение ниже этого порога, то значение максимального элемента max1 становится равным 0.
  \item Сравнивается значение полученного максимального элемента max1 и текущим максимальным значением max. Если первое больше, то максимальный элемент max становится равным max1.
  \item Параллельно этому процессу, каждый интервал времени t2 получается значение максимального элемента max, это значение переводится в цветовое представление, и значение максимального элемента становится\\равным~-1.
\end{itemize}

Фрагмент функции обработки звукового потока представлен в листинге~\ref{lst:node__signalProcessing}.

\lstinputlisting[style=ES6, caption={Фрагмент функции обработки звукового потока}, label={lst:node__signalProcessing}, linerange={33-56}, consecutivenumbers=true]{assets/listings/practical/Server/music.js}

Полный код записи и обработки звука представлен в приложении~\ref{lst:MUSIC__ALL}.