\subsubsection{Выбор платфорны для реализации}

Для реализации поставленной задачи была выбрана платформа Raspberry Pi. Ввиду своей распространённости и доступности, была выбрана модель Raspberry Pi 3 model B.

Системные характеристики Raspberry Pi 3 model B~\cite{rpi-site}:

\begin{enumerate}
  \item Микроархитектура -- Cortex-A53 (ARM v8).
  \item Частота центрального процессора -- 1,2 ГГц.
  \item Количество ядер -- 4.
  \item Объём оперативной памяти -- 1 ГБ.
  \item Количество контактов GPIO -- 40.
  \item Количество USB портов -- 4.
  \item Коммуникации -- Ethernet, Wi-Fi, Bluetooth.
\end{enumerate}

Операционная система RPI располагается на внешней microSD-карте, на которую можно записать один из дистрибутивов Linux. Официально с помощью установщика Raspberry Pi Imager можно установить одну из следующих операционных систем:

\begin{enumerate}
  \item Raspberry Pi OS.
  \item Ubuntu.
  \item Manjaro.
  \item RISC OS.
\end{enumerate}

Raspberry Pi OS является операционной системой, разработанной Rasp\-berry Pi Foundation. Эта система основана на Debian~\cite{rpi-site}. Одним из главных недостатков этой операционной системы является малая производительность при высокой загрузке системы. Является проприетарной операционной системой.

Ubuntu также является операционной системой, основанной на Debian, разрабатывается группой Canonical~\cite{ubuntu}. Главным недостатком данной системы также является малая производительность системы после <<чистой установки>>, то есть сразу после установки системы,  долгое время запуска операционной системы и невозможность работы с GPIO. Является проприетарной операционной системой.

RISC OS является операционной системой, разработанной специально для компьютеров, использующих центральный процессор архитектуры ARM~\cite{risc-os}. Редко используется ввиду малой популярности. Является проприетарной операционной системой.

Manjaro является операционной системой, основанной на Arch Linux. Производительность при высокой нагрузке является достаточной, чтобы выполнять несколько требовательных к ресурсам системы задач. Является операционной системой с открытым исходным кодом~\cite{manjaro}. Также Arch Linux и Manjaro обладают подробной документацией.
