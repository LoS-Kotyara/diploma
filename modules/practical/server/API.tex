\subsubsection{API}

Для предоставления программного интерфейса управления комплексом был выбран протокол WebSocket.

Для создания WebSocket-сервера, необходимо сначала создать HTTP сервер, необходимый для первичной установки соединения между клиентом и сервером.

Для создания HTTP-сервера используется стандартная библиотека Node http. При этом необходимо указать порт, который будет обрабатывать сервер. Также для настройки HTTP-сервера для WebSocket необходимо при получении запросов в качестве ответа отправлять ошибку и завершать HTTP соединение. Пример настройки HTTP-сервера для WebSocket приведён в листинге~\ref{lst:http__set}.

\lstinputlisting[style=ES6, caption={}, label={lst:http__set}, linerange={116-116,118-122,124-124}, consecutivenumbers=true]{assets/listings/practical/Server/server.js}

Затем, для создания самого WebSocket-сервера, необходимо использовать библиотеку, предоставляющую возможность установления WebSocket соединения. В качестве такой библиотеки была выбрана библиотека <<websocket>>.

При инициализации сервера с помощью библиотеки websocket, необходимо в качестве обязательного входного параметра передать созданный ранее HTTP-сервер. Также Можно указать другие параметры, определяющие поведение сервера при получении новых запросов на подключение, например, параметр <<autoAcceptConnections>>, определяющий, должен ли сервер автоматически принимать запросы, либо производить предварительную фильтрацию. Пример инициализации WebSocket-сервера представлен в листинге~\ref{lst:websocket__init}.

\lstinputlisting[style=ES6, caption={}, label={lst:websocket__init}, linerange={126-129}, consecutivenumbers=true]{assets/listings/practical/Server/server.js}

Затем необходимо описать, какие действия необходимо выполнять серверу при получении нового WebSocket-запроса. Для этого необходимо переопределить обработчик запроса по умолчанию. Минимальный обработчик WebSoc\-ket-запросов представлен в листинге~\ref{lst:websocket__req}.

\lstinputlisting[style=ES6, caption={}, label={lst:websocket__req}, linerange={132-132, 136-136, 149-149, 154-154}, consecutivenumbers=true]{assets/listings/practical/Server/server.js}

Полный код WebSocket-сервера представлен в приложении~\ref{lst:WEBSOCKET__ALL}.
