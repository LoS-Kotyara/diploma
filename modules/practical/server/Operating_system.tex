\subsubsection{Выбор операционной системы}

Исходя из полученных сведений об операционных системах, доступных для установки на Raspberry Pi, был выбран дистрибутив Manjaro.

В Manjaro по умолчанию используется ALSA в качестве звуковой подсистемы, поэтому для записи звуковых потоков используется утилита arecord, а для воспроизведения -- aplay.

Manjaro использует пакетный менеджер pacman, упрощающий управление пакетами. Для поддержания пакетов в актуальном состоянии, pacman синхронизируется с базами данных пакетов Manjaro посредством зеркал. Также возможна и установка пакетов <<вручную>> из исходных кодов.

Для начала разработки, необходимо установить все необходимые пакеты, такие как gcc, предоставляющую компиляторы языков C и C++, make, предоставляющую сборку утилит из исходных кодов, sudo, упрощающую выполнение команд от имени администратора, и других пакетов. Для установки этой группы пакетов, необходимо выполнить следующую bash-команду:

\begin{lstlisting}[style=ES6, language=bash]
  sudo pacman -Syy base-devel
\end{lstlisting}

При установке пакетов произойдёт полная синхронизация баз данных пакетов, так как был указан параметр <<yy>>.