\subsubsection{Язык программирования}

В качестве основного языка разработки комплекса был выбран язык JavaS\-cript. Для взаимодействия с операционной системой используется среда выполнения JS -- NodeJS (Node). Чтобы иметь возможность выполнять Node-скрипты, необходимо предварительно установить пакет <<nodejs>> с помощью менеджера пакетов pacman, либо с помощью менеджера версий nvm. Ввиду своей простоты, была произведена установка с помощью pacman.

\begin{lstlisting}[style=ES6, language=bash]
  sudo pacman -S nodejs
\end{lstlisting}

Вместе с установкой Node, производится установка Node Package Manager (npm), с помощью которого можно управлять Node-проектами и устанавливать Node-пакеты.

Для создания нового проекта используется npm. Для инициализации проекта, необходимо выполнить команду <<npm init -y>>, которая создаст в текущей директории файл package.json, в котором хранится вся информация о проекте: название, автор, текущая версия, команды управления, используемые пакеты, или зависимости, и другая информация.