\subsection{React Router}

Любому веб приложению необходимо иметь маршруты, так как пользователь должен видеть, в каком месте приложения он находится в данный момент времени. В основном пользователь видит свое место положение в адресной строке его браузера. Таким образом, можно сделать вывод, что приложение должно уметь сопоставлять определённый URL соответствующей ему страницы.

Сама по себе библиотека React не предоставляет возможность маршрутизации, но предоставляют сторонние библиотеки, например, React Router.

Основными группами компонентов библиотеки React Router являются~\cite{react-router}:

\begin{enumerate}
  \item Маршрутизаторы;
  \item Маршруты;
  \item Элементы навигации.
\end{enumerate}

Компоненты-маршрутизаторы представлены компонентами BrowserRouter и HashRouter~\cite{react-router}. Они определяют набор маршрутов и выполняют сопоставление запроса в адресной строке браузера маршруту. Когда какой-либо из маршрутов совпадает с URL запроса, то этот маршрут выбирается для обработки запроса.

Компоненты маршруты представлены компонентами Route и Switch~\cite{react-router}. Route является основным компонентом-маршрутом. Именно эти отрисовываются при совпадении запроса с маршрутом. Switch позволяет выбрать первый попавшийся дочерний Route. Без использования компонента Switch, Router может отобразить несколько Route, если они подходят URl запроса.

Элементы навигации представлены компонентами Link, NavLink и Redirect~\cite{react-router}. Они являются ссылками и используются для перенаправления на другой URL. При этом при нажатии на элементы страницы, отрисованные с помощью компонентов Link или NavLink происходит перерисовка страницы и изменение URL без обновления страницы. Когда необходима переадресация с одного маршрута на другой используется компонент Redirect.
