\conclusion

В наши дни невозможно представить себе жизнь без искусственных источников света. Эти источники света применяются повсеместно, как для простого освещения помещений, так и для придания окружению чего-то особенного.

Проекты, связанные с концепцией <<умный дом>>, являются одними из наиболее молодых и популярных направлений разработки. Одну из важнейших ролей в таких проектах является создание и использование искусственных источников света.

В ходе выполнения выпускной квалификационной работы был разработан программно-аппаратный комплекс для управления светодиодными лентами с помощью звука, а также веб-приложение для управления комплексом. Впоследствии разработанный программно-аппаратный комплекс был протестирован при помощи разработанного веб-приложения. В ходе тестов комплекс показал свою эффективность и корректность.

Преимуществом данной разработки является возможность управления всем программно-аппаратным комплексом используя веб-приложение. Есть возможность управления состояниями светодиодной ленты и микрофона, а также их параметрами: яркостью свечения светодиодов, скоростью эффекта бегущих из центра огней, нижней границей звучания и максимальным процентом зашумлённости.

Также был произведён анализ предметной области, получены знания о способах записи, обработки и воспроизведения звуковых потоков, способы модуляции сигналов, изучены особенности работы с одноплатными компьютерами Raspberry Pi, изучены способы клиент-серверного взаимодействия, а также средства для разработки пользовательских интерфейсов. 

Таким образом, цель и задачи выпускной квалификационной работы были выполнены.

%\begin{itemize}
%  \item Изучены и проанализированы характеристики и особенности работы %светодиодов, подходящих для реализации поставленной цели;
%  \item изучены способы работы со звуковыми потоками;
%  \item изучены и проанализированы характеристик платформ, подходящих %для реализации поставленной задачи;
%  \item изучены и проанализированы средства для разработки %пользовательских интерфейсов;
%  \item изучены и проанализированы протоколы связи, подходящих для %реализации поставленной задачи;
%  \item изучены особенности работы с одноплатными компьютерами %Raspberry Pi;
%  \item разработан программно-аппаратный комплекс, позволяющий %управлять адресными светодиодными лентами с помощью звука;
%  \item разработано веб-приложение для управления программно-аппаратным %комплексом;
%  \item произведено тестирование разработанного программно-аппаратного %комплекса для управления светодиодными лентами при помощи веб %приложения.
%\end{itemize}