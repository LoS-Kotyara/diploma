\intro

Невозможно себе представить современную жизнь без искусственных источников света. Эти источники продолжают играть важнейшую роль в технологическом прогрессе и применяются во всех сферах, например в сфере развлечений.

Сочетание различных световых сигналов может влиять на настроение человека, делая его более спокойным, активным, радостным или грустным. В купе со звуковыми сигналами, это может оказывать более сильное воздействие, что активно применяется в сфере развлечений, например, в проведении световых шоу, концертов, презентаций компаний и других областях.

Одним из основных видов искусственных источников света являются светодиоды, полупроводниковые приборы, создающие оптическое излучение при пропускании через него электрического тока в прямом направлении. Адресные же светодиоды, в свою очередь, являются подтипом светодиодов, ключевой особенностью которых является возможность управления каждым светодиодом, находящимся в ленте адресных светодиодов, по отдельности за счёт встроенного контроллера. Но также в работе с адресными светодиодами существует трудность, заключающаяся в необходимости более сложного управляющего контроллера, чем для обычных светодиодов.

Звуками, в свою очередь, также можно управлять. Получая на устройстве считывания некий звуковой поток, можно его проанализировать, получить набор его набор характеристик, и, преобразовав, трансформировать его в другой звуковой поток, либо, например, в световой поток.

Реализовать трансформацию звукового потока в световой можно на различных платформах: на персональных компьютерах, плат с микроконтроллерами , либо одноплатных компьютерах Raspberry Pi. Ввиду своих дороговизны, низкой энергоэффективности и избыточной производительности, не оптимально производить эту операцию на персональных компьютерах. Arduino являются подходящим вариантом для реализации такой трансформации, но, ввиду недостаточности производительности, также не являются оптимальным решением, если необходимо в рамках одной платы реализовать несколько проектов. Наиболее оптимальным является использование одноплатных компьютеров Raspberry Pi, которые объединяют в себе удобство в использовании персональных компьютеров, использование одного из дистрибутивов Linux в качестве операционной системы, а так же энергоэффективность и компактность плат с микроконтроллерами.

Актуальность работы заключается в том, что в различных сферах требуется использование света и музыки для достижения поставленной цели. В разработках, связанных с концепцией <<умный дом>>, например, световые сигналы дополняют играющую музыку, включенную пользователем.

Целью выпускной квалификационной работы является разработка и тестирование программно-аппаратного комплекса, позволяющего управлять адресными светодиодными лентами с помощью звука, а так же веб-приложения, позволяющего управлять комплексом.

Для достижения цели были поставлены следующие задачи:

\begin{itemize}
  \item Обзор и анализ характеристик и особенностей работы светодиодов, подходящих для реализации поставленной цели;
  \item изучение способов работы со звуковыми потоками;
  \item обзор и анализ характеристик платформ, подходящих для реализации поставленной цели;
  \item обзор средств для разработки пользовательских интерфейсов;
  \item обзор и анализ протоколов связи, подходящих для реализации поставленной цели;
  \item изучение особенностей работы с одноплатными компьютерами Raspberry Pi;
  \item разработка комплекса, позволяющего управлять адресными светодиодными лентами с помощью звука;
  \item разработка веб-приложения для управления программно-аппаратным комплексом;
  \item тестирование разработанного программно-аппаратного комплекса для управления светодиодными лентами при помощи веб приложения.
\end{itemize}